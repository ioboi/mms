% !TeX root = ./Serie03-JoelZuber-YannikDaellenbach.tex
% Benutzen Sie Normans Interaktionsmodell um folgende Handlung zu analysieren: 
% Ein Benutzer möchte eine Datei auf dem Schreibtisch seines Computers umbenennen 
% (z.B. von Document.txt zu Pruefungsfragen.txt). 
% Erläutern Sie anhand dieses Beispiels, die Begriffe Gulf of Execution und Gulf of Evalation.
\subsection*{Iteration 1}
\subsubsection*{Ziel}
Der Benutzer möchte, dass anstelle des existierenden Dateinamens das Textfeld zum Umbenennen der Datei angezeigt wird.
\subsubsection*{Planen}
Dazu muss er die Datei zweimal antippen, mit einer kurzen Wartezeit zwischen den beiden Klicks, damit sie nicht als schneller Doppelklick interpretiert werden.
\subsubsection*{Spezifizieren}
Zunächst muss er den Cursor über den Dateinamen bewegen, dann einmal die linke Maustaste betätigen, kurz warten und nochmals die linke Maustaste betätigen.
\subsubsection*{Ausführen}
Der Benutzer bewegt den Cursor und führt den langsamen Doppelklick aus.
\subsubsection*{Wahrnehmen}
Um den Dateinamen erscheint ein dünner Rahmen.
\subsubsection*{Interpretieren}
Der Benutzer folgert, dass sich das Textfeld geöffnet hat.
\subsubsection*{Vergleichen}
Dies ist das Textfeld, das er öffnen wollte.

\subsection*{Iteration 2}
\subsubsection*{Ziel}
Der Benutzer möchte alle Zeichen links vom Punkt (ohne Dateiende) markieren.
\subsubsection*{Planen}
Dies möchte er durch einen schnellen Doppelklick erreichen.
\subsubsection*{Spezifizieren}
Er will den Cursor auf die linke Seite des Punkts bewegen und zweimal schnell die linke Maustaste bedienen.
\subsubsection*{Ausführen}
Der Benutzer bewegt den Cursor und führt einen Doppelklick aus.
\subsubsection*{Wahrnehmen}
Der ausgewählte Teil des Dateinamens wird blau markiert.
\subsubsection*{Interpretieren}
Der Benutzer folgert, dass die markierte Zeichenfolge jetzt markiert ist.
\subsubsection*{Vergleichen}
Dies ist der Teil des Dateinamens, den er bearbeiten wollte.

\subsection*{Iteration 3}
\subsubsection*{Ziel}
Der Benutzer möchte den markierten Text durch neuen Text ersetzen.
\subsubsection*{Planen}
Dazu will er die Zeichenfolge mithilfe einer Tastatur eingeben.
\subsubsection*{Spezifizieren}
Er muss ein Zeichen nach dem anderen eingeben, indem er die jeweils entsprechende Taste auf der Tastatur drückt.
\subsubsection*{Ausführen}
Der Benutzer tippt auf der Tastatur herum.
\subsubsection*{Wahrnehmen}
Die eingegebenen Zeichen erscheinen nach jedem Tastenanschlag im Textfeld. Dahinter ist ein blinkender Strich.
\subsubsection*{Interpretieren}
Der Benutzer folgert, dass die neue Zeichenfolge registriert wurde. Ausserdem folgert er aus dem blinkenden Strich, dass die Bearbeitung des Dateinamens noch aktiv ist.
\subsubsection*{Vergleichen}
Der angezeigte Text entspricht dem vorgesehenen Dateinamen.

\subsection*{Iteration 4}
\subsubsection*{Ziel}
Der Benutzer möchte den Vorgang abschliessen und die Änderungen speichern.
\subsubsection*{Planen}
Dies möchte er durch Benutzung der \enquote{Enter}-Taste erreichen.
\subsubsection*{Spezifizieren}
Dazu muss er nur die \enquote{Enter}-Taste drücken.
\subsubsection*{Ausführen}
Der Benutzer drückt \enquote{Enter}.
\subsubsection*{Wahrnehmen}
Das Textfeld verschwindet und der neue Dateiname wird angezeigt. Zur Verstärkung des Feedbacks wird die Datei dabei markiert, was der Benutzer sofort wahrnimmt.
\subsubsection*{Interpretieren}
Die Änderung der Farbe der Dateivorschau weist den Benutzer unmittelbar darauf hin, dass etwas passiert ist. Er nimmt an, dass der Dateiname im System geändert wurde.
\subsubsection*{Vergleichen}
Der angezeigte Dateiname entspricht jenem, der für die Datei vorgesehen war.

\subsection*{Gulf of Execution}
Der grösste \textit{Gulf of Execution} in diesem Beispiel wird in Iteration 1 überwunden. 
Dank seiner Erfahrungen mit dem benutzten Betriebssystem in der Vergangenheit kennt der Benutzer eine schnelle und komfortable Möglichkeit, 
das Bearbeitungs-Textfeld zur Umbenennung von Dateien zu öffnen.

\subsection*{Gulf of Evaluation}
Ein grosser \textit{Gulf of Evaluation} ist in Iteration 4 enthalten. 
Wie kann sich der Benutzer sicher sein, dass sich im System wirklich etwas geändert hat? 
Dies wird durch ein starkes Feedback beim Betätigen der \enquote{Enter}-Taste überwunden und auch wieder von den Erfahrungen des Benutzers.