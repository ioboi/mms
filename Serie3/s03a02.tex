% !TeX root = ./Serie03-JoelZuber-YannikDaellenbach.tex
% Besprechen Sie den Zusammenhang zwischen natürlichem Mapping und der Scrollrichtung bei einer Maus.
Bei einer Maus muss man das Rad nach unten drehen, um nach unten zu scrollen. 
Dabei wird der Bildschirminhalt nach oben geschoben. 
Die Bewegung entspricht jener, 
wenn man ein physisches Blatt Papier unter einem fixierten Rad hat: 
Bewegt man die obere Seite des Rads nach unten, 
bewegt sich hinten die untere Seite des Rads nach oben und schiebt durch Reibung das Blatt nach oben, 
sodass der weiter unten angesiedelter Inhalt sichtbar wird.