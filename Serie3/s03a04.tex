% !TeX root = ./Serie03-JoelZuber-YannikDaellenbach.tex
% (a) Unterscheiden Sie Konzeptmodell, mentales Modell und System-Image. 
% (b) Erläutern Sie die Wichtigkeit des System-Images beim Design einer Interaktion.
\textbf{a)}\\\\
\begin{tabularx}{\linewidth}{|X|X|X|}
  \hline
  \textbf{Konzeptmodell} & \textbf{Mentales Modell} & \textbf{System-Image} \\
  \hline
  (Meist vereinfachte) Darstellung wie etwas funktioniert. & 
  Konzeptmodell im Gedächtnis der Menschen ~--~ Interpretation der Funktionsweise. &
  Gesamtheit aller Information die den Benutzenden über ein technisches Gerät zur Verfügung steht, inklusive gemachter Erfahrungen (etwa mit anderen Geräten).\\
  \hline
\end{tabularx}
\\\\
Aus dem System-Image entsteht das mentale Modell der Benutzenden.
Dieses mentale Modell sollte möglichst dem Konzeptmodell der Designerinnen und Designer entsprechen.
\\\\
\textbf{b)}
\\\\
Aus dem System-Image leiten die Benutzenden die Funktion des Geräts ab.
das System-Image fehlerhaft oder unvollständig können die Benutzenden
gar kein oder kein korrektes mentales Modell erstellen. 
Ohne ein korrektes mentales Modell kann das Gerät nicht korrekt genutzt werden.