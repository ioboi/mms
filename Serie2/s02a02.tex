% !TeX root = ./Serie02-JoelZuber-YannikDaellenbach.tex

% Weshalb darf man i.a. nicht davon ausgehen, 
% dass Benutzer etwas auf einem Computerbildschirm sehen, 
% nur weil es da ist 
% (insbesondere, wenn während der Interaktion eine Änderung auf der Oberfläche vorgenommen wird)?
Die selektive Wahrnehmumg des Menschen kann die Wahrnehumg grosser Änderungen im Gesichtsfeld verhindern.
Es ist nicht einmal sicher, dass die Benutzer alles sehen, was auf der Bildschirmoberfläche geschieht, 
insbesondere bei vielen kleinen Objekten.
Vor allem aber verhindert die selektive Wahrnehmung den Schluss 
\enquote{im Blickfeld $\Rightarrow$ gesehen}. Das Gehirn selektiert die Informationen, 
die es vom Auge erhält, sofort. Nur als wichtig eingestufte Informationen werden wahrgenommen. 
Es ist möglich, dass ein Objekt zwar im Blickfeld des Benutzers ist, 
sein Fokus aber gerade woanders liegt und es deshalb gar nicht wahrgenommen wird!