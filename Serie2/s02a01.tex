% !TeX root = ./Serie02-JoelZuber-YannikDaellenbach.tex

% Obwohl die Mitte des Bildschirms für das zentrale Sehen wichtig ist, 
% sollte man die periphere Sicht des Betrachters nicht ignorieren, 
% da selbst kleinste Änderungen im peripheren Sichtfeld sehr gut wahrgenommen werden.
%
% Erläutern Sie zwei Konsequenzen dieser Tatsache für die Gestaltung einer Applikationsoberfläche.
\textbf{Konsequenz 1} \\
Falls sich die Benutzenden konzentrieren müssen, können sie durch Aktivitäten im peripheren Sichtfeld
abgelenkt werden.
\\\\
\textbf{Konsequenz 2}\\
Neue Ereignisse können durch bewegliche Meldungen am Rand angezeigt werden,
ohne dass die Tätigkeit der Benutzenden unterbrochen werden muss. 
\\
Die Benutzenden werden die Meldung trotzdem wahrnehmen.