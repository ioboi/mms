% !TeX root = ./Serie02-JoelZuber-YannikDaellenbach.tex

% Sogenannte Hörsymbole (Auditory Icons) verwenden natürliche Klänge, 
% um verschiedene Arten von Objekten und Aktionen auf einer Oberfläche darzustellen. 
% Der SonicFinder für den Macintosh wurde aus dieser Idee heraus entwickelt, 
% um die Schnittstelle durch Redundanz zu verbessern. 
% Informieren Sie sich über den SonicFinder und beschreiben Sie Hörsymbole 
% für mindestens drei verschiedene Aktionen auf der Oberfläche.
Das Kopieren einer Datei ist, sobald es begonnen hat, 
ein laufender Prozess, der durch das imitierte Geräusch eines sich füllenden Glases als solcher für den Benutzer klarer erkennbar ist. 
Die Höhe des Geräusches gibt dabei den Fortschritt des Prozesses an, aber diese Information ist redundant zur sichtbaren Fortschrittsleiste.
Das Bewegen einer Datei in den Papierkorb erzeugt ein einzigartiges Geräusch,
dass den Benutzer an einen echten Papierkorb erinnert und das er schnell mit der \enquote{Zerstörung} einer Datei verbindet. 
So wird er sich zum Beispiel beim versehentlichen Löschen einer Datei eher bewusst, was er getan hat.
Der Zugriff auf einen externen Speicher wird mit einem metallischen Geräusch versehen,
dass die Anzeige auf dem Bildschirm nicht nur optisch, 
sondern auch akustisch als echtes Objekt erscheinen lässt. 
Die Tonhöhe richtet sich dabei nach der Grösse des Speichers, 
was den Benutzer intuitiv an grössere oder kleinere Objekte denken lässt. 
Auch hier ist die Information redundant, da die Grösse der Speicher auch in Zahlen angezeigt werden kann.