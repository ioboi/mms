% !TeX root = ./Serie02-JoelZuber-YannikDaellenbach.tex

% Erkennen vs. Erinnern:
% a) Was ist der Unterschied zwischen Erkennen und Erinnern in Bezug auf das menschliche Gedächtnis?
% b) Was kann ein Designer tun, um die Speicherlast eines Benutzers zu verkleinern?
% c) Kommentieren Sie folgende Aussage: 
%    Ein Menü sollte nicht mehr als 4 oder 5 Menü-Item beinhalten, da Menschen nur ca. 4 Chunks im Kurzzeitgedächtnis halten können.
\textbf{a)}\\
Der Unterschied zwischen Erkennen und Erinnern ist, beim Erkennen sind mehr Hinweise vorhanden als beim Erinnern. 
\\Erinnern ist deshalb der schwierigere Prozess.
\\\\
\textbf{b)}\\
Ein Designer kann die Informationen gruppiert darstellen, damit sich die Benutzenden diese einfacher merken können.
Das System sollte auch so gestalten sein, 
dass Benutzende die Informationen so bald wie möglich wiedergeben müssen, 
damit die Merkzeit 15 Sekunden nicht überschreitet und die Wahrscheinlichkeit einer Ablenkung minimiert wird. 
Idealerweise autorisiert der Designer das Programm, sich alle Informationen zu merken und so die Benutzenden zu entlasten.
\\\\
\textbf{c)}\\
Die Begrenzung der 4 Chunks im Kurzzeitgedächtnis bezieht sich auf das Erinnern.
In dieser Situation ist aber Erkennen gefragt: 
Die Benutzenden müssen sich später nicht an alle Menüpunkte erinnern.