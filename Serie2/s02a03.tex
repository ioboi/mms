% !TeX root = ./Serie02-JoelZuber-YannikDaellenbach.tex

% Wählen Sie ein beliebiges physisches Gerät aus, 
% das Sie perönlich verwenden 
% (z.B. Kaffeemaschine, Wecker, Fernbedienung, Abwaschmaschine, etc.). 
% Analysieren und Bewerten Sie die Anordnung und Organisation der Kontrollelemente/Anzeigen.

Beispiel Fernbedienung (Quickline):\\
Die Tasten auf der Fernbedienung sind in funktional logische Blöcke eingeteilt: (von oben nach unten) Ein-Ausschaltknöpfe, Zahlen, OK-Taste mit Pfeilen und Navigationstasten rundherum, farbige programmierbare Tasten, Lautstärke- und «Zapp»-Tasten, Tasten für Replay-Funktionen (umrahmt), spezielle weitere Tasten. Dies erleichtert es dem Benutzer, die Funktion von Tasten zu erkennen (zum Beispiel wenn diese Taste mit Replay zu tun hat, dann die daneben auch).
An- und Ausschalttasten sind farblich passend (grün und rot), aber auch angeschrieben. Das erleichtert das Verständnis, aber es ist auch für farbenblinde Personen möglich, die Tasten zu unterscheiden.
Die \enquote{Zapp}-Tasten sind rechts, die Lautstärke-Tasten links. Für Rechtshänder wird dadurch die Distanz von der Daumen-Ruheposition zu den häufiger benutzten \enquote{Zapp}-Tasten sehr klein, was sich gemäss Fitts' Gesetz positiv auf die benötigte Bewegungszeit auswirkt.
Die (häufig verwendeten) Navigationstasten sind sehr gross, was wiederum nach Fitts die Bewegungszeit verkürzt.
Allerdings sind diese Tasten nicht sehr intuitiv angeordnet. Insbesondere befinden sich \enquote{Back} und \enquote{Exit to TV} beide auf der rechten Seite, was gegen die sequentielle Anordnung verstösst, zu der das menschliche Gehirn im Zuge der Musterbildung tendiert: In der europäischen Kultur ist es üblich, dass zeitliche Abläufe von links nach rechts dargestellt werden. Entsprechend gehören Tasten wie \enquote{Back} und \enquote{Exit to TV} auf die linke Seite.
Die \enquote{unwichtigen} Tasten für spezielle Funktionen sind ganz unten auf der Fernbedienung nur wahrnehmbar, wenn man aktiv nach einer Taste sucht. 
Ausserdem ist das Betätigen dieser Tasten am umständlichsten, da entweder der Daumen stark verbogen oder die Fernbedienung in der Hand nach oben geschoben werden muss.

% TODO May create another example