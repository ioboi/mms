% !TeX root = ./Serie05-JoelZuber-YannikDaellenbach.tex
\begin{exercise}
  Nehmen Sie an, Sie wollen verstehen, wie Menschen USB Sticks bzw.Cloud-Lösungen zur Speicherung von Daten verwenden. 
  Was Sie besonders interessiert, ist wie junge Erwachsene und ältere Benutzer diese Speichermöglichkeiten nutzen.
  \begin{itemize}
    \item Skizzieren Sie eine mögliche Struktur einer Umfrage hierzu.
    \item Definieren Sie 5 Fragen, die Sie in einer Umfrage stellen könnten.
    \item Würden Sie Kontingentfragen verwenden? Wenn ja, wo?
    \item Wie würden Sie sicherstellen, dass Sie genügend Antworten erhalten?
  \end{itemize}
  \vspace{0.5cm}
\end{exercise}
\textbf{Struktur:}
\begin{itemize}
  \item Einleitung und Motivation: Thema Speicherung von Daten, anonym, 10 Minuten, alles beantworten, schöne Kreuzchen machen
  \item Fragen zu Einstellungen und Gewohnheiten
  \item Soziodemographische Fragen
  \item Danke, wo einreichen
\end{itemize}

\textbf{Fragen:}
\begin{itemize}
  \item Benutzen Sie USB-Sticks im Alltag zur Speicherung von Daten?
  \item Benutzen Sie Cloud-Lösungen im Alltag zur Speicherung von Daten?
  \item Wann haben Sie zuletzt einen USB-Stick im Berufsalltag zur Speicherung von Daten benutzt?
  \item Wann haben Sie zuletzt eine Cloud-Lösung im Berufsalltag zur Speicherung von Daten benutzt?
  \item Welches Medium würden Sie zur Speicherung von privaten Fotos vorziehen?
\end{itemize}
\textbf{Kontingentfragen:} Ja, im interessanten Teil der Umfrage. Z.B. Nutzen Sie Cloud-Lösungen? Wenn ja: Wie oft?
\\\\
\textbf{Genügend Antworten:} parallel Online- und Papier-Umfrage, auch zum Erreichen verschiedener Altersgruppen
