% Informieren Sie sich über Vannevar Bushs Publikation "AsWeMayThink" aus dem Jahr 1945.
% Bush entwirft darin unter anderem das Konzept des MEMEX.
% Beschreiben Sie dieses Konzept (in einem Paragraphen) und ordnen sie die Bedeutung dieses Konzeptes für heutige HCI Systeme ein.
Der MEMEX sammelt Informationen (Bilder, Zeitungen, ganze Bücher) in komprimierter Form und zeigt sie auf einem Bildschirm, wann immer man sie braucht. Er wird mit Tastatur, Knöpfen und Hebeln bedient, arbeitet sehr schnell und hat sehr viel Speicherplatz.
Bush beschreibt nicht nur die Funktionsweise, sondern auch die Benutzung des MEMEX. Dieser ist nicht nur für Spezialisten zugänglich, sondern auch für die breite Masse nützlich und nutzbar. Er ist leicht zu bedienen, merkt sich Positionen in Büchern und kann mehrere Artikel nebeneinander anzeigen. Es gibt sogar eine Art Suchfunktion, sodass ein benötigter Artikel nicht selbst gesucht werden muss, sondern nur sein Name eingetippt und die Maschine sucht für den Menschen.
1945 lag der Fokus der Forschung ganz klar auf den Funktionen und Möglichkeiten von Maschinen. Der Essay \enquote{As We May Think} ist ein gewaltiger Schritt in Richtung Usability-Forschung und Mensch-Maschine-Interaktion.
