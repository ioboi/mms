% !TeX root = ./Serie04-JoelZuber-YannikDaellenbach.tex
\begin{exercise}
  Informieren Sie sich über die 8 Goldenen Regeln von Ben Shneiderman 
  (eine weitere Sammlung von Richtlinien) und stellen Sie diese den zehn Heuristiken von Nielsen/Molich gegenüber. 
  Wo finden Sie Analogien, wo Unterschiede?
  \\\\
\end{exercise}
\begin{tabularx}{\textwidth}{|X|X|}
  \hline
  \textbf{Shneidermann} & \textbf{Nielsen/Molich}\\
  \hline
  Konsistenz & Konsistenz und Standards\\
  \hline
  Informatives Feedback & Sichtbarkeit des Systemstatus\\
  \hline
  Abgeschlossenheit & \\
  \hline
  Fehler vermeiden & Fehlervermeidung\\
  \hline
  Umkehrbarkeit & Benutzerkontrolle \& Freiheit\\
  \hline
  Benutzerkontrolle gewährleisten & Benutzerkontrolle \& Freiheit\\
  \hline
  Kurzzeitgedächtnis entlasten & Erkennen statt Erinnern\\
  \hline
  Universelle Benutzbarkeit & Minimales Design und Flexibilität und Effizienz\\
  \hline
  & Übereinstimmung Realität\\
  \hline
  & Fehler erkennen\\
  \hline
  & Hilfe und Dokumentation\\
  \hline
\end{tabularx}
\\\\\\
% Sichtbarkeit des Systemstatus
% Übereinstimmung zwischen System und der realen Welt
% Benutzerkontrolle und Freiheit
% Konsistenz und Standards
% Fehlervermeidung
% Erkennen statt Erinnern
% Flexibilität und Effizienz der Nutzung
% Ästhetisches und minimalistisches Design
% Anwendern helfen, Fehler zu erkennen, zu diagnostizieren und zu beheben
% Hilfe und Dokumentation
Die meisten Punkte stimmen überein oder fassen Punkte der Anderen zusammen.\\\\
Shneiderman nennt zusätzlich die \textit{Abgeschlossenheit}, 
welche bei Nielsen/Molich fehlt.
Im Gegensatz zu Nielsen/Molich fehlen bei Shneidermann die \textit{Übereinstimmung zwischen System und Realität}, 
\textit{Fehler erkennen} sowie \textit{Hilfe und Dokumentation} und er beschränkt sich auf \textit{Informatives Feedback}
während Nielsen/Molich eine ständige \textit{Sichtbarkeit des Systemstatus} verlangen.