% !TeX root = ./Serie04-JoelZuber-YannikDaellenbach.tex
\begin{exercise}
  Verwenden Sie zwei Grundsätze der Benutzerfreundlichkeit, 
  um eine \textit{Usability Spezifikation} für einen elektronischen Kalender zu erstellen.
  \begin{itemize}
    \item{
      Identifizieren Sie zunächst für beide Grundsätze einen typischen \textit{Use Case}, 
      der von einem Benutzer ausgeführt werden könnte.
    }
    \item{
      Vervollständigen Sie dann die \textit{Usability Spezifikation}, 
      wobei Sie davon ausgehen, 
      dass das elektronische System ein papierbasiertes System ablösen soll.
    }
  \end{itemize}
  \vspace{0.5cm}
\end{exercise}
\textbf{Use Case}:\\
Jemand erhält eine Einladung für eine Veranstaltung und möchte wissen,
ob bereits ein überschneidender Termin zu diesem Zeitpunkt geplant ist.
Er oder sie will also so schnell wie möglich genau wissen,
was an einem bestimmten Datum für Termine eingetragen sind.
Daraus ergeben sich die Grundsätze \enquote{schnelle Navigation} und \enquote{schnell erfassbare Anzeige der Informationen}.\\\\
\textbf{Messmethode:} Eine Testperson erhält die Aufgabe 
zu einem bestimmten Termin zu navigieren/den relevanten Bildschirminhalt wiederzugeben.
Dabei wird die benötigte Zeit in Abhängigkeit der Distanz zum Termin/der Menge des Bildschirminhalts gemessen.\\\\
\textbf{1st Level:} Auf einem grossen Papierkalender dauert es etwa zwei Sekunden einen Termin des aktuellen Monats zu finden,
jedoch locker zehn Sekunden,
wenn der Termin in einem anderen Monat ist (weil umgeblättert werden muss). 
Das Lesen der relevanten Informationen dauert etwa zwei Sekunden.\\\\
\textbf{Schlechtester Fall:} Zwölf Sekunden für die Navigation, fünf Sekunden zum Erfassen der Anzeige.\\
\textbf{Geplantes Level:} Drei Sekunden für die durchschnittliche Navigation, zwei Sekunden zum Erfassen der Anzeige.\\
\textbf{Bester Fall:} Zwei Sekunden für die Navigation, eine Sekunde zum Erfassen der Anzeige.