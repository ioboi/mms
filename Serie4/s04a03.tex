% !TeX root = ./Serie04-JoelZuber-YannikDaellenbach.tex
\begin{exercise}
  Betrachten Sie die folgende Schnittstelle zur Eingabe der Medikamentendosis in einem automatischen Transfusionssystem.
  Der Benutzer wollte eigentlich eine Dosis von 1372 eingeben. 
  Erörtern Sie kurz, 
  weshalb diese Fehleingabe nicht wirklich menschliches Versagen ist. 
  Wie könnten Sie die Schnittstelle verbessern, damit solche Fehler verhindert werden könnten?
  \\\\
\end{exercise}
Derartige Ausrutscher liegen in der Natur des Menschen und es ist die Aufgabe des Designers, 
ein System zu schaffen, welches mit diesen angemessen umgeht.
Die Schnittstelle kann verbessert werden, 
indem nach je drei Zahlen ein Abstand oder Strich angezeigt wird, z.B. \enquote{13 672} oder \enquote{13'672} statt \enquote{13672}.
So wäre die Interpretation der Zahl leicher.\\\\
Ausserdem: Wenn 1372 die empfohlene Menge des Medikaments ist, ist 13672 (fast zehnmal so viel) wahrscheinlich eine viel zu hohe Zahl.
Würde der Computer für gewisse Mengen Grenzwerte kennen, so könnte er den Arzt/die Ärztin warnen.