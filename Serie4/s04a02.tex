% !TeX root = ./Serie04-JoelZuber-YannikDaellenbach.tex
\begin{exercise}
  Einige der besprochenen Ausrutscher (Siehe Kapitel 5.4 im Skript, Aufzählung) können als Muster in einem State Transition Network (STN) sichtbar gemacht werden. 
  Skizzieren Sie mindestens drei Muster in einem STN, 
  die auf potentielles Auftreten eines Ausrutschers hinweisen können.
\end{exercise}

\begin{figure}[H]
  \centering
  \begin{tikzpicture}

    % defaults
    \tikzset{
      node distance=2cm,
      initial text=$ $
    }

    \node[state, initial](q1){};
    \node[state, right of=q1](q2){};
    \node[state, right of=q2](q3){};
    \node[state, right of=q3](q4){};
    \node[state, right of=q4](q5){};
    \node[state, above right of=q5](a){};
    \node[state, below right of=q5](b){};

    \draw [->] (q1) edge[above] node{} (q2);
    \draw [->] (q2) edge[above] node{} (q3);
    \draw [->] (q3) edge[above] node{} (q4);
    \draw [->] (q4) edge[above] node{} (q5);
    \draw [->] (q5) edge[above] node{} (a);
    \draw [->] (q5) edge[above] node{} (b);
  \end{tikzpicture}
\end{figure}

\begin{figure}[H]
  \centering

  \begin{tikzpicture}

    % defaults
    \tikzset{
      node distance=2cm,
      initial text=$ $
    }

    % stm mode 1
    \node[state, accepting](s1){};
    \node[state, right of=s1](s2){};
    \node[state, below right of=s1](s3){};
    \node[state, right of=s3](s4){};
    \node[state, below left of=s3](s5){};
    \node[state, right of=s5](s6){};

    % s1 edges
    \draw [->] (s1) edge node{} (s2);
    \draw [->] (s1) edge node{} (s3);
    
    % s2 edges
    \draw [->] (s2) edge node{} (s4);
    \draw [->] (s2) edge[bend right] node{} (s3);

    % s3 edges
    \draw [->] (s3) edge[bend right] node{} (s2);
    \draw [->] (s3) edge node{} (s4);
    \draw [->] (s3) edge node{} (s6);

    % s4 edges
    \draw [->] (s4) edge[below right] node{click} (s6);

    % s5 edges
    \draw [->] (s5) edge node{} (s3);

    % s6 edges
    \draw [->] (s6) edge node {} (s5);

    % mode 1 box
    \node [draw=black, fit= (s1) (s4) (s6)] (mode1) {};
    \node [above] at (mode1.north) {Mode 1}; 

    % stm mode 2
    \node[state, accepting, xshift=6cm] (t1){};
    \node[state, right of=t1](t2){};
    \node[state, below right of=t2](t3){};
    \node[state, left of=t3](t4){};
    \node[state, below left of=t4](t5){};
    \node[state, right of=t5](t6){};
    \node[state, accepting, right of=t6](t7){};

    % t1 edges
    \draw [->] (t1) edge node{} (t2);
    \draw [->] (t1) edge node{} (t4);

    % t2 edges
    \draw [->] (t2) edge node{} (t3);
    \draw [->] (t2) edge[bend right] node{} (t4);

    % t3 edges
    \draw [->] (t3) edge[right] node{click} (t7);

    % t4 edges
    \draw [->] (t4) edge[bend right] node{} (t2);
    \draw [->] (t4) edge node{} (t3);
    \draw [->] (t4) edge node{} (t6);

    % t5 edges
    \draw [->] (t5) edge node{} (t4);

    % t6 edges
    \draw [->] (t6) edge node{} (t5);

    \node [draw=black, fit= (t1) (t7)] (mode2){};
    \node [above] at (mode2.north) {Mode 2};

    \draw [->] ([yshift=0.5cm]mode1.east) to node[above]{change} ([yshift=0.5cm]mode2.west);
    \draw [->] ([yshift=-0.5cm]mode2.west) to node[below]{change} ([yshift=-0.5cm]mode1.east);

    % start arrow
    \node[left of=mode1, xshift=-1cm](start){};
    \draw [->] (start) to (mode1.west);

    \node[right of=mode2, xshift=1cm](end){};
    \draw [->] (mode2.east) to (end);

  \end{tikzpicture}
\end{figure}

\begin{figure}[H]
  \centering
  \begin{tikzpicture}

    % defaults
    \tikzset{
      node distance=2cm,
      initial text=$ $
    }

    \node[state, accepting](1){};
    \node[state, right of=1](2){};
    \node[state, right of=2](3){};
    \node[state, right of=3](4){};
    \node[state, right of=4](5){};

    \node[state, below of=5](6){};
    \node[state, left of=6](7){};
    \node[state, left of=7](8){};
    \node[state, left of=8](9){};
    \node[state, left of=9](10){};

    \node[state, below of=10](11){};
    \node[state, right of=11](12){};
    \node[state, right of=12](13){};
    \node[state, right of=13](14){};
    \node[state, accepting, right of=14](15){};

    %edges
    \draw[->](1) edge (2);
    \draw[->](2) edge (3);
    \draw[->](3) edge (4);
    \draw[->](4) edge (5);
    \draw[->](5) edge (6);
    \draw[->](6) edge (7);
    \draw[->](7) edge (8);
    \draw[->](8) edge (9);
    \draw[->](9) edge (10);
    \draw[->](10) edge (11);
    \draw[->](11) edge (12);
    \draw[->](12) edge (13);
    \draw[->](13) edge (14);
    \draw[->](14) edge (15);

    \node[draw=black, fit=(1) (15)](process){};
    \node [above] at (process.north) {Process};

    \node[left of=10](start){};
    \draw[->] (start) to (process);

    \node[right of=process, xshift=5cm, yshift=1cm](esc){};
    \draw[->] ([yshift=1cm]process.east) to node[above]{ESC}(esc);

    \node[right of=process, xshift=5cm, yshift=-1cm](end){};
    \draw[->] ([yshift=-1cm]process.east) to node[above]{FINISH}(end);
    
  \end{tikzpicture}
\end{figure}