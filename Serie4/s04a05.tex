% !TeX root = ./Serie04-JoelZuber-YannikDaellenbach.tex
\begin{exercise}
  Erötern Sie den Zusammenhang zwischen\ldots
  \begin{itemize}
    \item{
      \textit{deduktivem Denken} und der Heuristik \textit{\enquote{Übereinstimmung zwischen
      System und der realen Welt}} von Nielsen/Molich.
    }
    \item{
      \textit{induktivem Denken} und dem Designgrundsatz \textit{\enquote{Konsistenz}} 
      (oder der ISO Eigenschaft \textit{\enquote{Erwartungskonformität}}).
    }
    \item{
      \textit{abduktivem Denken} und der Designrichtlinie \textit{\enquote{Feedback}} von Norman.
    }
  \end{itemize}
  \vspace{0.5cm}
\end{exercise}

\begin{itemize}

  % Deduktives Denken: Schluss vom Allgemeinen auf das Besondere, die logisch notwendige Schlussfolgerung aus gegebenen Prämissen
  %
  % Übereinstimmung zwischen System und der realen Welt: 
  % Das System sollte die Sprache der Benutzer sprechen, 
  % mit Wörtern, Phrasen und Konzepten, die dem Benutzer vertraut sind, und nicht mit systemorientierten Begriffen. 
  % Folgen Sie den Konventionen der realen Welt 
  % und lassen Sie Informationen in einer natürlichen und logischen Reihenfolge erscheinen. (S.41)
  \item{
    Menschen lernen ihre Welt und deren allgemeine Regeln kennen und wenden diese auf spezifische Objekte in der Welt an.
    Das nennt man \enquote{deduktives Denken}. 
    Die Heuristik \enquote{Übereinstimmung mit der realen Welt} verlangt,
    dass Benutzende Regeln, die in der realen Welt gelten,
    gefahrlos auf Computersysteme deduzieren können.
  }

  % Induktives Denken ist die Verallgemeinerung (oder die Gewinnung von allgemeinen Aussagen)
  % aus der Betrachtung mehrerer Einzelfälle. (S.42)
  %
  % Konsistenz: Konsistenz bezieht sich auf die Ähnlichkeit im Verhalten,
  % die sich aus ähnlichen Situationen oder ähnlichen Aufgabenzielen ergibt. 
  % Konsistenz ist eines der am weitesten verbreitete Prinzipien. 
  % Konsistenz ist keine Eigenschaft eines interaktiven Systems,
  % die entweder erfüllt ist oder nicht. 
  % Stattdessen kann Konsistenz nur in Bezug auf eine Referenz angewendet werden. (S.116)
  \item{
    Menschen tendieren dazu, 
    aus Eigenschaften spezifischer Objekte auf universelle Regeln zu schliessen.
    Das nennt man \enquote{induktives Denken}.
    Der Designgrundsatz \enquote{Konsistenz} sowie die ISO-Eigenschaft \enquote{Erwartungskonformität} verlangen,
    dass es in einem System allgemein gültige Regeln gibt,
    die Benutzende lernen können.
  }

  % Abduktion schliesst von einer Tatsache zu der Handlung, 
  % die diese verursacht hat. 
  % Dies ist die Methode, mit der wir Erklärungen für die von uns beobachteten Ereignisse ableiten. (S.43)
  %
  % Feedback: 
  % Es gibt ausgiebige Informationen bezüglich der Folgen von Handlungen und des aktuellen Status des Produkts oder eines Dienstes. 
  % Nach der Ausführung einer Handlung ist der neue Status leicht zu bestimmen. (S.130)
  \item{
    Menschen versuchen,
    aus einem Ereignis auf ein verursachendes Ereignis zu schliessen.
    Das nennt man \enquote{abduktives Denken}. 
    Die Designrichtlinie \enquote{Feedback} verlangt,
    dass einer Benutzeraktion ein Ereignis folgt,
    aus dem Benutzende auf ein systeminternes Ereignis schliessen können.
  }
\end{itemize}