% !TeX root = ./Serie04-JoelZuber-YannikDaellenbach.tex
\begin{exercise}
  Was ist der Unterschied zwischen einem Ausrutscher und einem Irrtum? 
  Wie kann ein Designer das Auftreten von beiden minimieren?
  \\\\
\end{exercise}
% Ausrutscher passieren während der Ausführung des Plans, in der Wahrnehmung oder der Interpretation eines Ergebnisses — auf den niederen Ebenen der Kognition. (S.97)
% Irrtümer sind Fehler, die bei der Zielsetzung, der Aufstellung des Plans und beim Vergleich der Resultate mit den Erwartungen auftreten -- auf en höheren Ebene der Kognitiion. (S.96f)
In Normans Interaktionsmodell passieren Fehler in den Stadien Ziel,
Planen, Vergleichen und Ausrutscher in den Stadien Spezifizieren, Ausführen, Wahrnehmen und Interpretieren.\\
Ein Designer kann das Auftreten von Irrtümern minimieren, indem er klare Affordances, wo nötig Signifiers, gutes Mapping, Constraints, verständliches Feedback und ein verständliches Konzeptmodell verwendet.
Das Auftreten von Ausrutschern minimiert er, indem er auf wahrnehmbares Feedback achtet, das den veränderten Zustand beschreibt und \textit{Ungeschehen-Machen} ermöglicht.